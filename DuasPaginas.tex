\chapter{Introdução}

O crescimento constante das grandes cidades ao redor traz consigo diversos desafios relacionados à mobilidade urbana e gerência de veículos. A frota de veículos no mundo continua a crescer com o passar dos anos e vários problemas, das mais diversas naturezas, surgem desse aumento. É preciso então, cada vez mais, encontrar soluções inteligentes e eficientes para lidar com as situações que emergem. Programas de melhoria e estímulo ao transporte coletivo, redução de emissões de veículos e novas tecnologias de consumo de combustível têm sido eficazes em batalhar esses problemas.

Um problema de particular interesse para os moradores de grandes cidades é o da disponibilidade de vagas de estacionamento em espaços comercias e residenciais. No seu caminho para o trabalho um morador de um grande centro metropolitano pode passar uma boa parte do tempo de seu trajeto circulando estacionamentos procurando um espaço disponível para estacionar seu carro. Além de perder tempo, o motorista consume mais combustível do que o desejado e seu veículo emite poluentes na atmosfera. É fácil ver então os benefícios que um sistema que facilite essa procura por vagas traria.

Não só usuários de estacionamentos se beneficiaram de tal sistema. Um negócio que possua um sistema facilitador para o estacionamento instalado atrai novos clientes, que apreciam não ter que passar pela árdua tarefa de procura de vagas. Uma boa logística para estacionamentos pode ser um fator que diferência um estabelecimento dos seus concorrentes e o coloca acima dos demais. Uma solução eficiente para estacionamentos é então, portanto, uma questão que vai além do conforto. É uma questão de economia, lucro, competitividade e sustentabilidade.

Já existem soluções e tecnologias elaboradas para solucionar, ou ao menos amenizar, esse problema, porém atualmente elas são no geral muito caras ou pouco eficazes. Soluções como sensores nas entradas e saídas de estacionamentos são capazes apenas de informar a ocupação atual do estacionamento aos motoristas, enquanto sensores individuais em cada vaga tornam a implementação cara e de difícil escalabilidade. Diversas alternativas já foram exploradas com algum sucesso, como detalhadas em \cite{idris09}. Esse trabalho tem particular interesse nas soluções que se aproveitam de processamento de imagem para determinar ocupação do estacionamento e diponibilidade individual de cada vaga.

Utilizar processamento de imagens tem diversas vantagens. É necessário apenas a instalação de um pequeno número de câmeras em posições estratégicas, barateando o custo inicial e o custo de manutenção. O sistema também é pouco intrusivo, não sendo necessárias obras para fazer a sua instalação e futuras obras não afetariam o desempenho do sistema em quase nada. Por outro lado, existem desafios exclusivos a esses tipos de sistema. A mudança da iluminação com o passar do dia, ruído na imagem da câmera e encontrar uma posição apropriada são problemas reais. Em particular estacionamentos com tetos baixos podem não ser adequados para a implementação de um sistema desse tipo. No geral porém, utilizar processamento de imagens é uma alternativa eficiente e pouco custosa, principalmente para estacionamentos abertos que possuem altos postes de luz.

Esse \textit{paper} tem como objetivo propor uma solução para os desafios encontrados em tais sistemas. Ao final do trabalho terá sido apresentada uma solução para amenizar o problema da mudança de iluminação, uma técnica para a detecção das vagas e a sua ocupação e maneiras de se adquirir informações sobre a permanência dos veículos.

O primeiro problema a ser resolvido é o de detectar a entrada e saída de veículos no estacionamento. Para isso, o programa proposto detecta o carros em movimento e analisa a direção desse movimento para determinar se a ocupação do estacionamento aumentou ou não. É possível acompanhar os veículos e determinar a região geral aonde eles estacionaram, caso tenham encontrado uma vaga. Isso possibilita determinar informação mais precisa sobre a ocupação do estacionamento. Se o programa detecta que um carro parou em uma vaga em um corredor específico, ele é capaz de dizer então quantas vagas livres há em cada uma das colunas.

Para detectar corretamente o movimento dos carros no estacionamento é preciso ser capaz de identificar o asfalto e outros elementos estáticos na imagem, ou seja, segmentar os veículos do fundo. É importante definir corretamente a imagem de \textit{background} para que se posso aplicar técnicas de segmentação dos veículos nas vagas(\cite{delibaltov2013parking}\cite{true2007vacant}) com precisão. Porém, as características do fundo não são completamente estáticas. Na medida que a iluminação natural muda durante o dia e fontes de iluminação artificial se acendem, a tonalidade dos elementos estáticos na imagem muda. Além disso, outros fatores como clima podem influenciar na imagem e fazer com que ela se torne consideravelmente diferente do seu estado anterior e gere falsos positivos. A fim de combater esse problema, o trabalho conta com um sistema de geração dinâmica de \textit{background}. Existem vários métodos bem sucedidos para se concluir essa tarefa\cite{chen2012dynamic}\cite{hai2009self}\cite{shoushtarian2003practical}. O método escolhido para esse trabalho é parecido com o apresentado em \cite{hai2009self}.

Esse método se utiliza da diferença entre dois quadros consecutivos para identificar uma região aonde ocorreu movimento. Essa região é então expandida a fim de garantir que ela contenha o objeto em movimento desejado. Uma vez identifica essa região, o programa opera sobre o quadro atual e as informações da diferença entre os quadros para atualizar a imagem de \textit{background} com a informação da região onde não houve movimento. Assim, o programa é capaz de gerar um fundo adaptativo. Objetos que param de se movimentar se integram ao fundo e a iluminação na imagem de \textit{background} muda junto com a da imagem obtida da câmera. Essa implementação será discutida com mais detalhes posteriormente.

