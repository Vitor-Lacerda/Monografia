\chapter{Considerações finais}

O programa ainda está em um estado inicial, no qual ainda estão sendo desenvolvidos componentes básicos para a implementação do programa em si e os testes estão sendo feito sobre funções específicas. Nessa etapa o foco foi em adquirir o conhecimento necessário para a implementação final e explorar opções que podem vir a ser usadas no programa definitivo. Várias técnicas foram estudadas e testadas, várias ainda estão sendo. Ainda há um elemento de exploração de possibilidades nessa etapa da implementação do trabalho.

Durante esse tempo de testes, concluiu-se que implementar um projeto completo com o escopo indicado é definitivamente viável, tanto técnicamente quanto conceitualmente. A abordagem tomada até o momento também parece estar na direção correta. Sendo assim, o projeto continuará no caminho que está sendo tomado atualmente, com previsão de término em pouco tempo.

As etapas futuras de implementação envolvem mais testes em imagens artificais, criadas especificamente para o trabalho, como fotografias e animações manipuladas através de programas de edição de imagens, simulando diversas condições de ambiente, luz e clima para que seja feita uma pré-validação das funcionalidades separadamente e depois como um todo antes de se iniciarem os testes em com imagens reais e em tempo real.

Espera-se que ao final desse trabalho, tenha-se produzido um \textit{software} que atende a todos os objetivos apresentados e que seja capaz de ajudar diversas pessoas de formas distintas e importantes, sendo uma pequena parte da solução de um problema que afeta tanto motoristas comuns quanto donos de estabelecimento comerciais e até metrópoles globais como um todo. Ao término do programa, espera-se também que tenha acontecido ainda mais aprendizado na área de Processamento de Imagens, que continua se mostrando cada vez mais importante no mundo comercial.

 